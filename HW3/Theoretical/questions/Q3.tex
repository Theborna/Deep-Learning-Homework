\subsection{سوال 3}

در این تمرین قصد داریم به بررسی دو شبکەی معروف یعنی Networks Convolutional Connected Densely\cite{huang2018densely}
و \lr{U-Net}\cite{ronneberger2015unet} بپردازیم. برای بررسی هرچه بهتر این دو شبکه بهتر است به لینک های زیر مراجعه نمایید.


\begin{enumerate}
	\item {
	      \textbf{سوالات مفهومی مربوط به U-Net :}
	      \begin{itemize}
		      \item {
		            ویژگی اصلی شبکەی U-Net که آن را از یک شبکه کانولوشنی عادی متمایز می دارد چه می باشد و دلیل اینکه ما شاهد یک ساختار U شکل هستیم چه می باشد؟

		            \begin{qsolve}[]
			            ساختار U شکل این شبکه اصلی ترین عامل تمایز بین این شبکه و شبکه های کانولوشنی معمولی است،
			            این ساختار شامل 3 قسمت اصلی و \lr{skip connection} بین آنها است که دقت این شبکه را برای کار های
			            segmentation بسیار بالا میبرد.
			            \begin{itemize}
				            \item {
				                  \textbf{مسیر با کاهش بعد (Encoder) :} شامل چندین لایه کانولوشنی با پولینگ که مهم ترین feature ها را استخراج میکنند.
				                  }
				            \item {
				                  \textbf{Bottleneck : } قسمت میانی و نازک که دارای مهم ترین feature ها یاد گرفته شده توسط Encoder است.
				                  }
				            \item {
				                  \textbf{مسیر با افزایش بعد (Decoder) :} این قسمت با استفاده از داده های \lr{skip connection} و چندین لایه upconv و up-sampling تصویر را بزرگ کرده تا \lr{segmentation map} نهایی را بسازد.
				                  }
			            \end{itemize}
			            حالت U شکل این ساختار کمک میکند تا اطلاعات لوکال و گلوبال را حفظ کنیم.
		            \end{qsolve}
		            }
		      \item {
		            می دانیم که connection Skip ها نقشی پررنگ در این شبکەها دارند، دلیل حضور این مورد را در
		            شبکەی U-Net بیان کنید.

		            \begin{qsolve}[]
			            این اتصالات به ما کمک میکند تا اطلاعات مکانی و دیگر اطلاعاتی که ممکن است در لایه های down-sampling و encode کردن
			            از دست رفته باشد را باز بدست آوریم، پس به ساختار کمک میکند تا اطلاهات high-level و low-level را با هم ادقام کنیم تا کار دقیق تری انجام دهیم.
		            \end{qsolve}
		            }
		      \item {
		            چرا این نوع از اتصالات در تصاویر پزشکی دارای
		            اهمیت بیشتر می باشند و چه کمکی به ما در دامنەی تشخیص موارد پزشکی می کنند؟

		            \begin{qsolve}[]
			            این شبکه به طور خاص برای عملیات segmentation طراحی شده است، اتصالات خاص این نوع معماری همراه با \lr{skip connection} ها
			            به ساختار کمک میکند که برای مثال در تشخیص تومور، با ادقام اطلاعات سطح بالا و اطلاعات مکانی و سطح
			            پایین، مرز تومور ها را به دقت تشخیص دهند و چیز های غیرمعمول لوکال را تشخیص دهند.

			            دقت بالای این سیستم نیز آن را کاندید خوبی برای کار های پزشکی میکند.
		            \end{qsolve}
		            }
	      \end{itemize}
	      }
	      \clearpage
	\item {
	      \textbf{سوالات محاسباتی مربوط به U-Net :}
	      \begin{itemize}
		      \item {
		            تصور کنید که ابعاد تصویر ورودی ما برای این شبکه ۲۵۶ × ۲۵۶ می باشد. حال فرض می شود که در
		            این معماری هرلایه در انکدر ابعاد را به نصف کاهش می دهد و در دیکدر دو برابر می کند. در پایین ترین
		            لایه (عمیق ترین لایه) این معماری، فضای ویژگی ما چند پیکسل خواهد داشت؟

		            \begin{qsolve}[]
			            \begin{minipage}[r]{0.49\linewidth}
				            در ساختار U-Net دارای 4 لایه انکدر هستیم، در نتیجه:
				            \begin{multline*}
					            256^2\to128^2\to64^2\to32^2\to16^2
				            \end{multline*}
				            ولی با تعداد $1024$ کانال اطلاعات خواهیم بود، با در نظر گرفتن کاهش ابعاد ناشی از کانولوشن ولی:
				            \begin{multline*}
					            256^2\to252^2
					            \to126^2\to122^2\\
					            \to 61^2\to57^2
					            \to28^2\to24^2\\
					            \to12^2\to8^2
				            \end{multline*}
			            \end{minipage}
			            \begin{minipage}[l]{0.5\linewidth}
				            \includegraphics*[width=\linewidth]{pics/U-Net.png}
			            \end{minipage}
						که برای سایز خروجی باید در واقع این عدد را ضرب در $c\times c$ نیز کرد، که برای لایه آخر $c=1024$. 
		            \end{qsolve}
		            }
		      \item {
		            در U-Net، فرض کنید انکودر دارای لایەهایی با ،۶۴ ،۱۲۸ ۲۵۶ و ۵۱۲ فیلتر است. اگر هر لایه
		            کانولوشن از کرنال های ۳ × ۳ استفاده بکند، تعداد پارامترهای لایه کانولوشن دوم انکدر را محاسبه
		            کنید.

		            \begin{qsolve}[]
			            لایه دوم شامل دو لایه کانولوشنی با فیلتر های یکسان و یک لایه پولینگ است، لایه پولینگ پارامتر قابل یادگیری ندارد، ولی
			            لایه های کانولوشنی هر کدام شامل 128 فیلتر $3\times3$ میباشند، به تعداد کانال نیز باید دقت شود، که تعداد پارامتر در نتیجه:
			            \[
				            \text{parameters} = 128\times(3\times3)\times64+128+128\times(3\times3)\times128+128=221,440
			            \]
		            \end{qsolve}
		            }
	      \end{itemize}
	      }
	\item {
	      \textbf{سوالات مفهومͳ DenseNet : }
	      \begin{itemize}
		      \item {
		            تفاوت های اصلی بین \lr{ResNet's residual connections} و \lr{DenseNet's Dense connections} را بیان
		            کنید. درمورد هرکدام از موارد گفته نیز، توضیح مختصری بدهید.

		            \begin{qsolve}[]
			            در ResNet ما از \lr{Skip connection} استفاده میکنیم که یک یا چند لایه مشخص را رد میکنند، به صورتی که انگار تابع نهایی به صورت
			            $H(x)=F(x)+x$ است، و ما قسمت غیرخطی را محاسبه میکنیم، در DenseNet ما از \lr{Dense connection} استفاده میکنیم،
			            به ایت صورت که هر لایه، خروجی تمامی لایه های قبل خود را دریافت میکند که انگار به صورت $H(x)=H_l([x_0,x_1,\dots,x_{l-1}])$ میباشد.

			            \begin{itemize}
				            \item {
				                  \textbf{ResNet :} در این ساختار ما از چندین \lr{Residual block} استفاده میکنیم که هماتطور که گفته شد دارای \lr{skip connection} هستند،
				                  این نوع اتصالات هم به generalization کمک کرده هم به شارش گرادیان کمک میکند.
				                  }
				                  \item{
				                              \textbf{DenseNet : } این ساختار از چندین \lr{Dense block} استفاده میکند که با دادن اطلاعات تمامی لایه های پیشین به ورودی هر لایه،
				                              ساختار را تشویق به استفاده دوباره و چند باره از فیچر ها، بهبود انتقال اطلاعات و بهبود شارش گرادیان در backpropagation میکند.
				                        }
			            \end{itemize}
		            \end{qsolve}
		            }
		      \item {
		            بیان کنید که DenseNet چگونه مشکل \lr{vanishing gradient} را کاهش میدهد و مزیت آن چه می باشد؟

		            \begin{qsolve}[]
			            ساختار DenseNet مشکل گرادیان را اینگونه بهتر میکند که با استفاده از \lr{Dense connection}، مسیر های کوتاه تری
			            برای گرادیان وجود دارد که \lr{backpropagation} را انجام دهد، و در عمل انگار خروجی هر لایه و پارامتر های هر لایه یک
			            مسیر مستقیم تا اطلاعات \lr{loss function} دارند که به یادگیری بهتر جفت فیچر های low-level و high-level کمک میکند.

			            مزیت های دیگر این است که طبق اینکه سیستم به اطلاعات لایه های قبل دسترسی دارد و آنها را دوباره استفاده میکند،
			            ساختار به این تشویق میشود که فیچر های جدید تر و مفید تری یاد بگیرد بجای یادگیری فیچر های تکراری و بی استفاده.

			            همچنین این استفاده دوباره از فیچر ها از کم شدن دقت با افزایش زیاد عمق که ساختار های معمولی دچار آن هستند جلوگیری میکند و از
			            اشباع شدن سیستم نیز جلوگیری میکند.
		            \end{qsolve}
		            }
	      \end{itemize}
	      }
	\item {
	      \textbf{سوالات محاسبەای DenseNet : }

	      \begin{itemize}
		      \item {
		            در یک DenseNet با سه لایه در یک Block Dense اگر لایه اول ۶۴ فیچر مپ تولید کند، لایه دوم
		            ۱۲۸ فیچر مپ و لایه سوم ۲۵۶ فیچر مپ تولید کند، لایه سوم چند فیچر مپ ورودی را دریافت خواهد
		            کرد؟

		            \begin{qsolve}[]
			            \begin{minipage}[r]{0.49\linewidth}
				            یک لایه در DenseNet در واقع concatenate شده خروجی تمامی لایه های قبل خود را میگیرد،
				            در نتیجه میتوان به صورت زیر نوشت.
				            \begin{gather*}
					            x_3 = H_3([x_1, x_2])\\
					            |[x_1,x_2]| = |x_1|+|x_2|=64+128=192\\
				            \end{gather*}
				            در نتیجه 192 فیچر مپ ورودی داریم.
			            \end{minipage}
			            \begin{minipage}[l]{0.5\linewidth}
				            \includegraphics*[width=\linewidth]{pics/DenseNet.png}
			            \end{minipage}
		            \end{qsolve}
		            }
		      \item {
		            با در نظر گرفتن نرخ رشد k در DenseNet، اگر هر لایه k فیچر مپ جدید تولید کند و ورودی یک
		            block dense دارای ۳۲ کانال باشد، اگر ۲۴ = k باشد لایه سوم در بلوک چند کانال خروجی خواهد
		            داشت؟

					\begin{qsolve}[]
						در شبکه DenseNet پارامتر k توصیف کننده تعداد فیچر مپی است که هر لایه به ورودی لایه بعد اضافه میکند، 
						محاسبه به صورت زیر است.

						\[
							C_{out} = \underbrace{\overbrace{\overbrace{C_{in}}^{\text{input}} + k}^{\text{\lr{first layer}}} + k}_{\text{\lr{second layer}}}+k=C_{in} + k\times n\Rightarrow 32 + 72 = 104 
						\]

						در نتیجه 104 کانال خروجی خواهیم داشت.
					\end{qsolve}
		            }
	      \end{itemize}
	      }
\end{enumerate}