\subsection{سوال 1}

یک مسئله دستەبندی با 5 دسته، که دیتاست، شامل تصاویری به اندازه $10\times10$ پیکسل میباشند، داریم. دو
شبکه عصبی یک لایه را به صورت زیر در نظر بگیرید. توضیح دهید کدام یک انتخاب بهتری میباشد؟

\begin{itemize}
    \item {
        یک لایه connected fully که ورودی آن، flatten (بردار شده) تصاویر دیتاست میباشد.
    }
    \item {
        یک لایه کانولوشن که که در آن ۵ فیلتر به اندازه $10\times10$ داریم.
    }
\end{itemize}

\begin{qsolve}[]
    در این ساختار، این 2 شبکه با هم معادل اند.
    زیرا هر کدام از این فیلتر ها، سایزی برابر با سایز عکس ورودی دارند، در نتیجه هر کدام از این 
    فیلتر ها به مانند یک perceptron عمل میکنند.

    در نتیجه گویا 5 perceptron به صورت \lr{fully connected} بین عکس و خروجی داریم که به طور کامل معادل شبکه
    \lr{fully connected} عمل میکند.
    
    ولی به طور کلی شبکه های کانولوشنی برای این نوع استفاده ها بهتر عمل میکنند، ولی در این ساختار و با این ابعاد ورودی و 
    فیلتر، معادل شده با یک شبکه \lr{fully connected} میشوند.
\end{qsolve}