\section{سوال 2}

فرض کنید دیتاستی داریم شامل تصاویر رنگی به اندازه ۱۲۸ × .۱۲۸ میخواهیم یک شبکه عصبی کانولوشن
برای آن طراحی کنیم.

\begin{itemize}
	\item {
	      اندازه خروجی و تعداد پارامترهای لایه اول کانولوشن را محاسبه کنید اگر ۱۶ فیلتر ۵× ۵
	      با ۱ = stride و ۲ = padding داشته باشیم.
	      }

	      \begin{qsolve}[]
		      \begin{minipage}[l]{0.69\linewidth}
			      عکس با استفاده از فیلتر $3\times3$ است و ورودی های $5\times5$ ولی برای شهود کمک کننده است.

			      \begin{align*}
				      \text{parameters}       & =16\times5\times5+16=416          \\
				      \text{\lr{output size}} & =16\times(128+2\cdot2-(5-1))^2 \\
				                              & =16\times128\times128=262,144
			      \end{align*}
		      \end{minipage}
		      \begin{minipage}[r]{0.3\linewidth}
			      \includegraphics*[width=\linewidth]{pics/conv-pad-stride.png}
		      \end{minipage}
	      \end{qsolve}
	\item {
	      فرض کنید هر لایه، ۳ قسمت شامل : کانولوشن، pooling max و تابع فعالسازی (ReLU) را دارا
	      می باشد، که لایه های کانولوشنی، هر کدام شامل ۱۶ فیلتر ۵× ۵ با ۱ = stride و ۲ = padding و
	      لایەهای pooling max همگی ۲ × ۲ با ۲ = stride هستند. ۳ لایه با این مشخصات را پشت سر
	      هم در نظر بگیرید. اندازه تنسور در لایه خروجی نهایی و تعداد پارامترهای این ۳ لایه را حساب کنید.

	      \begin{qsolve}[]
		      همانطور که مشاهده شد، لایه های کانولوشن با سایز $5\times5$ با $padding=2$ به صورت \lr{same size} کار میکنند، حال کافیست لایه های
		      max-pooling را در نظر بگیریم.

		      لایه های max-pooling با \lr{stride=2, $2\times2$} به این صورت اند که ابعاد ورودی را نصف میکنند. پس سایز هر لایه به صورت زیر خواهد بود.
		      \begin{multline*}
			      16\times128\times128\xrightarrow{\text{conv}}16\times128\times128\xrightarrow{\text{pooling}}16\times64\times64\\
			      \xrightarrow{\text{conv}}16\times64\times64\xrightarrow{\text{pooling}}16\times32\times32\\
			      \xrightarrow{\text{conv}}16\times32\times32\xrightarrow{\text{pooling}}16\times16\times16
		      \end{multline*}

		      پس خروجی سایز $16\times16\times16$ را دارد.

		      برای تعداد پارامتر نیز، تعداد پارامتر فیلتر ها را باید شمرد، که هر لایه کانولوشنی تعداد $16\times5\times5$ پارامتر دارد.

		      \[
			      \text{parameters}=3\times16\times5\times5+16=3\times400+16=1216
		      \]
	      \end{qsolve}
	      }
	\item {
	      فرض کنید هدف، حل یک مسئله classification، که شامل ۱۰ دسته هست،می باشد.تعداد کل پارامترهای
	      شبکه را در این حالت حساب کنید.

	      \begin{qsolve}[]
		      ساده ترین کاری که میتوان کرد، این است که صرفا خروجی فیچر های بدست آمده از سوی لایه های کانولوشنی را
		      با استفاده از یک لایه خطی، به حالت طبقه بندی با softMax در آوریم. در این صورت:

		      \[
			      \text{classifier parameters} = 16\times16\times16\times10 + 10 = 40970
		      \]

		      \splitqsolve
		      که در این صورت کل پارامتر ها مقدار 42170 پارامتر خواهد شد، برای کاهش پارامتر،  میتوانیم در این لایه آخر نیز pooling انجام دهیم.
		      برای مثال اکر از \lr{Global Average pooling} در سطح کانال استفاده کنیم، تغداد پارامتر به صورت زیر خواهد بود

		      \[
			      \text{classifier parameters}_{GAP} = 16\times10 + 10 = 170
		      \]

		      که تعداد کل پارامتر را به 1370 تقلیل میدهد، که البته احتمالا برای این مسئله کم باشد.
	      \end{qsolve}
	      }
	\item {
	      در این قسمت، با یک مفهوم مهم آشنا می شویم. field Receptive بیانگر این است که نورون خروجی،
	      تحت تاثیر چه مقدار از نورون های ورودی می باشد. در حقیقت تعیین می کند هر نورون خروجی از چه
	      ناحیەای با چه اندازەای از تنسور ورودی تاثیر می پذیرد. حال field Receptive را برای یک نورون
	      خروجی لایه سوم ( قبل از لایه connected fully ( بررسی کنید. (برای فهم بهتر این مفهوم میتوانید به
	      این \href{https://blog.mlreview.com/a-guide-to-receptive-field-arithmetic-for-convolutional-neural-networks-e0f514068807?gi=9a91fd9bda2d}{لینک} مراجعه کنید)

	      \begin{qsolve}[]
		      \includegraphics*[width=\linewidth]{pics/receptive.png}
		      طبق اینکه داریم $s=1$ برای تمام لایه ها، در هر لایه خواهیم داشت:
		      \[
			      r_{out}=r_{in}+(k-1),\quad k:\text{kernel size}
		      \]
		      در نتیجه برای \lr{receptive field} لایه سوم خواهیم داشت:
		      \[
			      r_3 = 1 + 3\times(5-1)= 1 + 12 = 13
		      \]
		      پس یک قسمت $13\times13$ از عکس اصلی را میبیند.
	      \end{qsolve}
	      }
\end{itemize}